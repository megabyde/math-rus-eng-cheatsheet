\documentclass[a5paper, 10pt, twoside, numbers=enddot]{scrartcl}

\defaulthyphenchar=127
\usepackage{cmap}
\usepackage[T2A, T1]{fontenc}
\usepackage[utf8]{inputenc}

\usepackage[english, russian]{babel}
\usepackage{indentfirst}

\usepackage[left=1cm, right=1cm, top=1.5cm, bottom=2cm]{geometry}
\usepackage{microtype}

\usepackage{setspace}
\onehalfspacing

\setlength{\footskip}{1 cm}
\setlength{\parindent}{1 cm}

\usepackage{enumerate, xspace, calc}

\usepackage[
  unicode,
  colorlinks,
  linkcolor=blue,
  citecolor=red,
  bookmarksnumbered=true,
  pdftitle={},
  pdfauthor={},
  pdflang = {ru-RU},
  pdfpagelayout = {SinglePage}
]{hyperref}

\renewcommand{\theenumii}{\asbuk{enumii}}
\renewcommand{\dot}{\textbullet\xspace}
\newcommand{\cor}{\textit{or}\xspace}
\newcommand{\word}{$\sim$}

\newcommand{\eg}{\textit{e.\,g.,}\xspace}
% Et cetera (etc.) = И так далее (и т.д.)
\newcommand{\etc}{и~т.\,д.\xspace}
% Id est (i.e.) = То есть (т.е.)
\newcommand{\ie}{т.\,е.\xspace}
% Et alii (et al.) = И другие (и др.)
\newcommand{\etal}{и~др.\xspace}

\newcommand{\dictchar}[1]{
  \goodbreak\vspace{\baselineskip}%
  \LARGE\textbf{#1}\normalsize
  \par\nopagebreak
  \vspace{0.35\baselineskip}
}

\newcommand{\dictentry}[2]{
  \hspace{1.2em}%
  \parbox[t]{\textwidth-1.4em}{\raggedright\hspace{-1.4em}{\textbf{#1}}{ #2}}%
  \vspace{0.35\baselineskip}
  \par%
}

% ----------------------------------------------------------------------------------
\begin{document}

\begin{enumerate}
  \item При переводах необходимо использовать американский вариант правописания: например,
    <<analyze>>, а не <<analyse>>; <<analog>>, а не <<analogue>>; <<modeling>>, а не <<modelling>>;
    <<formulas>>, а не <<formulae>>; \etc

  \item Употребление артиклей: помните те немногие случаи, когда употребление того или иного артикля
    (или отсутствие артикля) обязательно, среди них:
    \begin{itemize}
      \item Фраза <<такой (такая, такое) \ldots>> при переводе требует неопределенного артикля (нет
        артикля во множественном числе) после слова <<such>>.\\ [4pt]
        \textsf{
          Такая функция обязательно существует.\\
          Such a function necessarily exists.\\ [4pt]
          Такие функции обязательно существуют.\\
          Such functions necessarily exist.
        }

      \item Сравнительной степени прилагательного в роли определения должен; как правило,
        предшествовать неопределенный артикль.\\ [4pt]
        \textsf{
          Для решения задачи необходим более эффективный алгоритм.\\
          For the problem to be solved a more effective algorithm is required.
        }

      \item Превосходной степени прилагательного в роли определения, должен предшествовать
        определенный артикль.\\ [4pt]
        \textsf{
          Наиболее эффективная процедура была разработана в работе [1].\\
          The most efficient procedure was developed in [1].
        }

      \item Обратите внимание на отсутствие артикля перед словом <<formula>>, за которым следует ее
        номер. В данном случае слову <<formula>> было присвоено <<имя собственные>> "--- ее номер,
        поэтому употребление артикля (определенного или неопределенного) избыточно. Сюда же
        относятся слова Section, Table, Figure (Fig.), Equation (Eq.) \etc\\ [4pt]
        \textsf{
          Формулой (1) исчерпываются все возможные случаи.\\
          Formula (1) exhausts all possible cases.
        }

      \item Никогда не ставьте артикль перед словом <<another>>. Ситуация со словом <<other>> "---
        более сложная: а) в значении <<какие-то другие>>, которое аналогично употреблению слова
        <<another>>, определенный артикль перед, <<other>> никогда не ставится; б) в значении
        <<другие (другой)>> (не тот, о котором речь шла выше), перед словом <<other>> всегда
        ставится определенный артикль, например: <<We had two pencils. I took one pencil, the other
        pencil is for you>>. <<The system comprises three equations. The first equation is similar
        to Eq. (1.1), while, the other two equations reflect peculiarities of the physical system
        being considered.\\ [4pt]
        \textsf{
          Другой случаи был рассмотрен в [1].\\
          Another case has been considered in [1].
        }

      \item Перед порядковыми числительными употребляется определенный артикль.\\ [4pt]
        \textsf{
          Второе уравнение в системе (3) не имеет решения.\\
          The second equation in system (3) has no solution.
        }

      \item \textsf{С вышеупомянутой задачей связан ряд (несколько) приложений.\\
        A number of applications are connected with the foregoing problem.\\ [4pt]
        \textbf{но}\\ [4pt]
        Число приложений, связанных с вышеупомянутой задачей, велико.\\
        The number of applications connected with the foregoing problem is large.
      }
    \end{itemize}

  \item Употребляйте слово <<rather>> в негативном контексте, в позитивном контекст употребляйте
    слово <<quite>>, например: <<The convergence of the series is rather bad>>, но <<The convergence
    of the series is quite good>>.

  \item Обратите внимание на употребление предлогов времени: by, at, since, till; during, in, for.
    Первые четыре предлога (by, at, since, till) относятся к определенному моменту времени, а
    предлоги <<in>>, <<during>>, <<for>> относятся к временному промежутку.
    \begin{enumerate}
      \item <<by>> отвечает на вопрос <<к какому времени>>, например:\\ [4pt]
        \textsf{
          By the time the processor finishes the calculation of the matrix, another processor will
          have already started finding the solution to the system.\\
          К тому моменту, когда процессор закончит вычисление матрицы, Другой процессор Уже начнет
          искать решение системы.\\ [4pt]
          The research is to be finished by August, 1992.\\
          Исследования должны быть завершены к августу 1992 г.
        }

      \item <<at>> отвечает на вопрос <<в какой момент времени>>:\\ [4pt]
        \textsf{
          At this, time moment a transition occurs from one state to another.\\
          B этот момент времени происходит переход из одного состояния в другое.
        }

      \item <<since>> отвечает на вопрос <<начиная с какого времени>> и всегда употребляется (в этом
        значении, а не в значении <<поэтому>>) с Present (или Past) Perfect:\\ [4pt]
        \textsf{
          Since that time scientists have been using this method.\\
          С тех пор ученые используют этот метод.\\ [4pt]
        }
        В высокопарном стиле можно вместо <<since>> использовать словосочетание <<ever since>>.

      \item <<till>> отвечает на вопрос <<до какого времени>>:\\ [4pt]
        \textsf{
          Till the discovery of neutrons a <<sandwich>> model of nucleus had been used.\\
          До открытия нейтрона использовалась протон-электронная модель ядра.
        }

      \item <<in>> отвечает на вопрос <<за (через) какое время>>:\\ [4pt]
        \textsf{
          The computation can be finished in an hour.\\
          Расчеты могут быть закончены за (через) один час.\\ [4pt]
          I will  be back in five minutes.\\
          Я вернусь через пять минут.\\ [4pt]
        }
        Обратите внимание, что то же самое можно сказать, используя оборот <<it will take \ldots do
        smth.>>, например, <<It will take an hour to compute the matrix>>.

      \item <<during>> отвечает на вопрос <<когда>> или <<во время чего>> и используется для
        указания на то, что два процесса происходят одновременно, с акцентом на одновременность:\\ [4pt]
        \textsf{
          During the calculation of the matrix other processors can compute the free term of the
          system.\\
          Во время вычисления матрицы другие  процессоры могут вычислять свободный член системы.\\ [4pt]
        }
        Распространенной ошибкой является употребление предлога <<during>> вместо предлога <<for>>.

      \item <<for>> отвечает на вопрос <<в течение какого времени (на какое время)>>, с акцентом на
        продолжительность временного промежутка:\\ [4pt]
        \textsf{
          I will go out for half an hour.\\
          Я отойду на полчаса.\\ [4pt]
          Trajectories can remain near it for a time period of order unity.\\
          Траектории могут оставаться около него в течение промежутка времени порядка единицы.
        }
    \end{enumerate}

  \item Оборот <<известно, что \ldots>> характерен, для русской научной лексики, в то  время как
    оборот <<it is known that \ldots>>, который является дословным переводом, в английской научной
    лексике употребляется редко. Обычно в предложении, которое содержит данный оборот, содержится
    указание на то откуда известно, например, ссылка на работу. Если это так, то в переводе
    необходимо указать авторов данной работы и строить на этом перевод предложения, например:\\ [4pt]
    \textsf{
      Известно [13], что эти операторы имеют простые собственные числа.\\
      Smith and Vensori [13] showed that these operators have simple eigenvalues.
    }

  \item Использование существительного в роли определения. Следует избегать употребления
    существительного в роли определения, если основное слово  обозначает действие над данным
    существительным, например: не следует  переводить <<решение системы>> Как <<the system
    solution>>, правильный перевод <<the solution of system>>.\\ [4pt]
    Избегайте слишком длинных (более трех определений) определительных рядов. Если без этого не
    удается обойтись "--- объединяйте члены определительного ряда, которые связаны друг с другом,
    при помощи дефиса, например <<linear-equation system>>.\\ [4pt]
    С другой стороны, не используйте <<of>>, если использование существительного в роли определения
    оправдано, например, <<the convergence rate>>, а~не <<the rate of convergence>> или <<the
    reaction temperature>>, а~не <<the tem\-perature of reaction>>.\\ [4pt]
    He используйте существительное в роли определения, если за ним следуют слова или выражения,
    которые по смыслу или грамматически связаны с~данным существительным, а не с определяемым
    словом, например, фраза <<температура реакции, описанной в [1]>> должно переводиться как <<the
    temerature оf reaction described in [1], а не <<the reaction temperature described in [1]>>.

  \item Никогда не используйте <<of>> после герундия перед прямым дополнением "--- это грубая
    грамматическая ошибка. Чтобы лучше запомнить это правило, можно провести следующую аналогию с
    русским языком: <<of>> в русском языке соответствует родительному падежу, в то же время прямое
    дополнение) после деепричастия в русском языке всегда стоит в винительном падеже (а не в
    родительном!) "--- <<делая кого, что>> $\rightarrow$ <<doing smth.>>, а не <<doing of smth.>>.
    Тем не менее, герундиальные формы глаголов часто используются в английском языке в качестве
    существительных (например, processing, testing \etal). Если такое употребление герундиальной
    формы в языке устоялось, то, в  соответствии с обычными правилами, после него можно ставить
    <<of>>, например, <<processing of experimental data>>.\\ [4pt]
    Никогда не используйте герундий в роли существительного, если существительное с тем же смыслом в
    языке уже существует, например: <<employment of this representation>>, а не <<employing of this
    representation>> или <<solution of the system>>, а не <<solving of the system>>.

  \item Делайте различие между <<may>> и <<can>>. <<May>> обозначает моральную категорию (\ie кто-то
    разрешил это сделать, это позволено делать), в~то время как <<can>> обозначает физическую
    возможность что-то сделать, осу\-ществимость действия, выражаемого основным глаголом. Употребляйте
    <<may>> с глаголами, обозначающими мыслительную деятельность "--- think, believe, consider,
    assume \etc

  \item Глаголы <<allow>>, <<permit>>, <<enable>> употребляются только в соответствии со следующими
    схемами:\\ [4pt] \textsf{
      allow smb. to do smth.\\
      allow smth. to be done\\
      allow smth.\\ [4pt]}
    Употребление <<allow to do smth.>> не допускается. Если возникают трудности с построением одной
    из конструкций, приведенных выше, всегда можно употребить безличное <<one>>, например:\\ [4pt]
    \textsf{
      This equation altows one to calculate the value of $f(x)$.\\ [4pt]
      \textbf{или}\\ [4pt]
      This equation allows the value of $f(x)$ to be calculated.
    }

  \item Используйте в переводе следующие сокращения:
    \begin{center}
      \sffamily
      \begin{tabular}{lll}
        equation  & $\rightarrow$ & Eq.\\
        equations & $\rightarrow$ & Eqs.\\
        figure    & $\rightarrow$ & Fig.\\
        figures   & $\rightarrow$ & Figs.\\
      \end{tabular}
    \end{center}
    если за словом следует номер, например, <<Eq. (1)>>, но <<this equation>> (обратите внимание на
    обязательный пробел после точки в сокращении). Если сокращаемое слово стоит и начале предложения,
    то оно не сокращается, например:\\ [4pt]
    \textsf{
      This is easily seen from Eq. (1).\\ [4pt]
      \textbf{но}\\ [4pt]
      Equation (1) clearly shows that.
    }

  \item Имена собственные:
    \begin{enumerate}
      \item В названиях теорем, равенств, неравенств \etc, образованных от имен собственных (фамилии
        ученых) придерживайтесь следующего правила: употребляйте притяжательный падеж, если перед
        словосочетанием нет артикля, и не употребляйте притяжательный падеж, если артикль
        (определённый или неопределённый) имеется, например:\\ [4pt]
        \textsf{
          Cauchy's operator\\
          the Cauchy operator
        }

      \item Обороты, образованные по схеме <<прилаг. (или сущ.) + по + фамилия>> "--- например,
        <<непрерывный по Липшицу>>, следует переводить как <<name + adjective (or noun.)>>,
        например, <<Lipschitz continuous>> или <<Lyapunov stability>> (<<устойчивость по
        Ляпунову>>).
    \end{enumerate}

  \item Перевод слова <<именно>>. Слово <<именно>> употребляется в русском языке в двух различных
    контекстах.
    \begin{enumerate}
      \item Для того, чтобы подчеркнуть, что данное действие было выполнено этим, а не каким-либо
        другим агентом, например:\\ [4pt]
        \textsf{Именно этот эффект обуславливает такое поведение системы.}\\ [4pt]
        В данном контексте данный оборот переводится следующим образом:\\ [4pt]
        \textsf{It is this phenomenon that accounts for such a behavior of the system.}

      \item Для указания начала расшифровки предыдущего члена предложения. В данном контексте слово
        <<именно>> чаще всего употребляется вместе с союзом <<а>> "--- <<а именно>>, например:\\ [4pt]
        \textsf{
          Данный эффект объясняется двумя причинами, а именно, большим градиентом температуры и
          плохой теплопроводностью стенок.\\
          This effect is accounted for by two reasons, namely large temperature gradient and a poor
          heat conductivity of the walls.\\ [4pt]
        }
        При переводе данного контекста, слова <<а именно>> можно просто опускать, заменяя их на
        двоеточие или тире; или использовать латинское выражение <<viz.>>, которое является точным
        синонимом <<namely>>.
    \end{enumerate}

  \item Согласование числа надлежащего и сказуемого.
    \begin{enumerate}
      \item Помните имена существительные, которые формально имеют множественное число, но должны
        употребляться со сказуемым в единственном числе, если речь идет о данных объектах, взятых и
        совокупности, а не по-отдельности:
        \begin{center}
          \sffamily
          \begin{tabular}{lllll}
            knowledge & news     & money  & information & contents\\
            dozen     & none     & range  & couple      & group\\
            number    & series   & data   & major       & pair\\
            variety   & progress & advice & hair        & fruit
          \end{tabular}
        \end{center}
        Например:\\ [4pt]
        \textsf{
          The series is arranged in order of decreasing size.\\ [4pt]
          \textbf{но}\\ [4pt]
          A series of experiments were performed.
        }

      \item Единицы измерения считаются собирательными существительными и употребляются со сказуемым
        в единственном числе, например:\\ [4pt]
        \textsf{
          To test the effect, 5 g of the substance was taken.\\ [4pt]
          Ten meters or a copper wire  serves as a heat conductor.\\ [4pt]
          Two years is needed to complete the experiments.
        }

      \item Составное подлежащее, содержащее слова <<each>>, <<every>>, могут употребляться со сказуемым в единственном числе, например:\\ [4pt] \textsf{
      Every value of $n$ is to be tested separately.}
    \end{enumerate}

  \item Использование дефиса.
    \begin{enumerate}
      \item Префиксы, перечисленные ниже, не отделяются дефисом при употреблении с нарицательными
        существительными и прилагательными, но обязательно отделяются дефисом при использовании с
        именами собственными.
        \begin{center}
          \sffamily
          \begin{tabular}{llllll}
            after    & de       & iso      & non      & pseudo  & trans\\
            ante     & di       & metallo  & over     & re      & ultra\\
            anti     & down     & mid      & photo    & semi    & un\\
            auto     & electro  & macro    & physico  & sub     & under\\
            bi       & extra    & micro    & poly     & up      & visco\\
            bio      & hyper    & mini     & post     & stereo  &\\
            co       & hypo     & mono     & pre      & super   &\\
            counter  & infra    & multi    & pro      & supra   &\\
          \end{tabular}
        \end{center}
        Например:
        \begin{center}
          \sffamily
          \begin{tabular}{lll}
            premultiplied  & а не & pre-multiplied\\
            multigrid      & а не & multi-grid\\
            nonposistive   & а не & non-positive\\
            antisymmetric  & а не & anti-symmetric\\
            cooperation    & а не & cooperation
          \end{tabular}
        \end{center}
        Ho: <<pre-Newtonian era>> или <<non-Gaussian distribution>>

      \item Не отделяйте дефисом суффикс <<like>>, если только это не приводит к~появлению тройной
        буквы <<l>>, например: <<chainlike>>, но <<ball-like>>.

      \item Не отделяйте дефисом суффикс <<fold>>, если только ему не предшествует цифра, например
        <<tenfold increase>> и <<multifold>>, но <<25-fold reduction>>.

      \item Не отделяйте дефисом суффикс <<wide>>, например: <<worldwide>>.

      \item Используйте дефис для разделения префикса и химических названий, например:
        <<non-hydrogen>>.

      \item Не используйте дефис, если первое из сокращаемых слов заканчивается на <<-lу>>, например
        <<recently developed method>>.

      \item Используйте дефис для разделения числа и единицы измерения, если они используются в
        качестве составного определения, например: <<a~15-minute-exposure>>, <<a 20-g sample>>, <<a
        20-m pipe>>.

      \item Если два и более модификатора связаны с одним и тем же основным словом "--- используйте
        дефис после каждого модификатора, но не повторяйте основное слово, например: <<first- and
        second-order equations>>, <<high-, medium-, and low-frequency measurements>>.

      \item Используйте дефис для отделения таких модификаторов, как <<well>>, <<ever>>, <<still>>,
        например: <<well-known scientist>>, <<ever-present danger>>, <<still-new equipment>>. Однако,
        дефис не используется, если данный модификатор  используется вместе с ещё одним модификатором,
        например: <<very well studied hypothesis>>.

      \item Не используйте дефис, если модификатор является собственным именем, например: <<Fourier
        transform technique>>.

      \item Не используйте дефис, если первое слово является сравнительной или превосходной степенью
        прилагательного, например: <<higher tempera\-tu\-re transition>>. Исключения: <<least-square
        analysis>>, <<nearest-neighbor interaction>>.

      \item Используйте дефис в модификаторах, которые содержат имена числительные, например:
        <<three-dimensional equation>>, <<two-phase system>>.

      \item Используйте дефис в модификаторах, которые содержат глаголы или глагольные формы,
        например: <<laser-induced reaction>>, <<problem-solving abilities>>, <<immobilized-phase
        method>>.

      \item Используйте дефис в модификаторах, которые состоят более, чем из двух слов, а также в
        модификаторах типа <<число-единица-измерения-слово>>, например: <<2-m-long pipe>>,
        <<3-year-old child>>, <<signal-to-noise ratio>>, <<out-of-plane distance>>.

      \item Используйте дефис в составных прилагательных, используемых в~качестве Predicate
        adjectives, например: <<This equation is first-order>>, <<This effect is
        temperature-dependent>>.
    \end{enumerate}

  \item Использование запятой.
    \begin{enumerate}
      \item В перечислительных рядах, содержащих три и более членов ставьте запятую перед <<and>>
        или <<or>> (включая библиографию и заголовок), например <<Ivanov, Petrov, and Kuliev [3]
        showed that \ldots>>, но  <<Smith and Venson [13] showed that \ldots>> или <<electrons,
        protons, and neutrons>>, но <<electrons and protons>>.

      \item Не ставьте запятую перед <<et al.>>, если ему предшествует только одна фамилия,
        например: <<Jones et al.>>, но <<Brown, Smith, et al.>>.

      \item Выделяйте запятыми обороты <<that is>>, <<for example>>, <<namely>>, <<viz.>>,
        <<e.\,g.>>, <<i.\,e.>>.

      \item Выделяйте запятыми придаточные предложения, которые вводятся словами <<which>>,
        <<where>> и <<who>>, например: <<The setup, which is shown in Fig. 1, is intended for
        \ldots>>, <<The equation $f(x) = a_{11}\log N(e)$, where $a_{11}$ is а factor, can be solved
        numerically>>.

      \item Отделяйте запятой длинные вводные фразы, например: <<Because of an increasing amount of
        water in the substance, it is not surprising to observe intense hydrogen lines in the
        spectrum>>.
    \end{enumerate}

  \item Перевод слова <<достаточно>> используемого в качестве модификатора.
    \begin{enumerate}
      \item Если словосочетание <<достаточно + полная форма прилагательного>> используется в
        качестве определения, то данный оборот следует переводить как <<sufficiently small +
        adjective>>, например, <<a sufficiently small value of the parameter>>.

      \item Конструкцию <<достаточно + краткая форма прилагательного>> следует переводить как в
        следующем примере: <<If the value of this parameter is small enough equation (1) can be
        rewritten as \ldots>>.

      \item Конструкции <<достаточно + краткая форма прилагательного + для того, чтобы>> следует
        переводить как в следующем примере: <<If the value of $N$ is large enough to ensure, that
        \ldots>>.

      \item Перевод существительных с модификаторами, указывающими их параметры. Обороты типа
        <<кабель длиной 10 м>> или <<отверстие диаметром 5 мм>> следует переводить как (обратите
        внимание на употребление артикля и предлога <<of>>):\\ [4pt]
        \textsf{
          a cable with a length of 10 m\\
          a 10-m-long cable\\
          a hole with a diameter of 5 mm\\
          a hole 5 mm in diameter\\ [4pt]
        }
        Неопределенный артикль в примерах, приведенных выше, может меняться на определенный в
        зависимости от контекста.\\ [4pt]
        Однако, если число вместе с единицей измерения имеет буквенное обозначение, например,
        <<резистор с сопротивлением $R=10$ Ом>> следует переводить как <<a resistor with
        resistance $R=10\ \Omega$>>. Обратите внимание на отсутствие артикля и отсутствие предлога
        <<of>>.
    \end{enumerate}

  \item Использование деепричастных оборотов. При использовании деепричастных оборотов, необходимо,
    чтобы было ясно, кто выполняет действие, выражаемое деепричастием (обычно это должно быть
    подлежащее основного предложения), например:\\ [4pt]
    \textsf{
      Using the procedure described previously, we can evaluate the partition function.\\ [4pt]
      \textbf{или}\\ [4pt]
      The partition function can be evaluated by using the procedure described previously.\\ [4pt]
      \textbf{но не}\\ [4pt]
      Using the procedure described previously, the partition function can be evaluated.\\ [4pt]
    }
    Обратите внимание на отсутствие предлога <<by>> в первом предложении (действие, выражаемое
    деепричастием, выполняется подлежащим основным глаголом) и на его наличие во втором предложении.

  \item Избегайте <<цветистости>>, многословия и наукообразия, например:
    \begin{center}
      \sffamily
      \begin{tabular}{lll}
        owing to the fact that      & $\rightarrow$ & because\\
        subsequent to               & $\rightarrow$ & after\\
        on the order of             & $\rightarrow$ & about\\
        in the near future          & $\rightarrow$ & soon\\
        at the present time         & $\rightarrow$ & now\\
        by means of                 & $\rightarrow$ & by\\
        it appears that             & $\rightarrow$ & apparently\\
        of great importance         & $\rightarrow$ & important\\
        in consequence of this fact & $\rightarrow$ & therefore\\
        a very limited number of    & $\rightarrow$ & few\\
        in spite of the fact that   & $\rightarrow$ & although
      \end{tabular}
    \end{center}

  \item Не пишите <<don't>>, а пишите <<do not>> и т.\,п., \ie используйте полные формы.

  \item Не путайте <<it's>> и <<its>>. <<It's>> "--- это краткая форма от <<it is>>, a <<its>> "---
    это <<его, ее>> по отношению к неодушевленным предметам.

  \item Не забывайте, что <<cannot>> пишется вместе.
\end{enumerate}

% ----------------------------------------------------------------------------------
\newpage
\setlength{\parindent}{0pt}
\singlespacing

\dictchar{А}
\dictentry{абсолютно непрерывна} {absolutely continuous;}
\dictentry{автоколебания} {self-oscillation;}
\dictentry{автоматически выполняться} {to be automatically satisfied;}
\dictentry{алгебра} {algebra;\\
  \textbf{внешняя \word} exterior algebra;\\
  \textbf{\word\ Хопфа} Hopf algebra;
}
\dictentry{аналитический} {analytic;}
\dictentry{аппроксимация} {approximation;\\
  \textbf{\word\ Бузинеска} Boussinesq approximation;
}

% ----------------------------------------------------------------------------------
\dictchar{Б}
\dictentry{без потери общности} {with no loss of generality;}
\dictentry{безусловный} {unconditional;}
\dictentry{бесконечномерный} {infinite-dimensional;}
\dictentry{близкий} {close, proximate;\\
  \textbf{\word е числа} proximate numbers;
}
\dictentry{быть в состоянии ч.-л. сделать} {to be able to do smth., to be capable of doing smth.;}
\dictentry{быть готовым к ч.-л.}{to be ready for smth. (\cor doing smth.),
  to be in position to~do smth. (\eg Now we are in a position to formulate the final result);
}

% ----------------------------------------------------------------------------------
\dictchar{В}
\dictentry{в дальнейшем} {in the sequel (\cor in the following);}
\dictentry{включительно} {inclusive (\eg its $x$-derivatives up to the second order inclusive);}
\dictentry{вложение} {imbedding, inclusion;\\
  \textbf{\word\ в целом} global imbedding;\\
  \textbf{отображение \word я} inclusion map;
}
\dictentry{вложенный} {imbedded;}
\dictentry{вне области} {in the exterior of the domain;}
\dictentry{внутри области} {in the interior of the domain;}
\dictentry{возникать} {arise;\\
  \textbf{\word ет естественный вопрос} the question naturally arises of \ldots;
}
\dictentry{вполне непрерывный} {completely continuous (compact, etc.);}
\dictentry{в противном случае} {otherwise (\eg We assume that $\|\hat{\varphi}\| > 0$
  [otherwise (2.16) is not needed in the proof]);}
\dictentry{Вронскиан} {Wronskian;}
\dictentry{вспоминать} {recall, а не remind;}
\dictentry{встречать препятствия} {encounter obstacles;}
\dictentry{входить в уравнение как параметр} {enter equation as a parameter;}
\dictentry{выколотый} {punctured (\eg a punctured neighborhood not containing eigenvalues of $B^m$);}
\dictentry{выражение} {expression;\\
  \textbf{\dot \word\ в квадратных скобках} expression in square brackets (\cor bracketed expression);
  \textbf{\dot \word\ в круглых скобках} expression in parentheses (\cor parenthesized expression);
  \textbf{\dot \word\ в фигурных скобках} expression in curly braces;
  \textbf{\dot \word\ под знаком интеграла} integrand;
  \textbf{\dot \word\ под знаком радикала} radicand;
}
\dictentry{выше} {in the foregoing, in the above;
  (\eg In the foregoing we defined matrix $S$ to be a product of two matrices.);\\
  \textbf{\dot как и выше} as in the foregoing;
}
\dictentry{выполняться (об уравнении, неравенстве, теореме \etc)} {hold (\cor be valid);}
\dictentry{вышеупомянутый} {foregoing;\\
  \textbf{\dot при \word\ условиях} under the foregoing conditions;
}

% ----------------------------------------------------------------------------------
\dictchar{Г}
\dictentry{гиперболоид} {hyperboloid;\\
  \textbf{усеченный однолистный \word\ вращения} truncated one-sheeted hyperboloid of revolution;
}
\dictentry{граница} {boundary;\\
  \textbf{\dot внешняя \word} exterior boundary;
}
\dictentry{грань} {face, side, bound;\\
  \textbf{нижняя \word} the greatest lower bound;\\
  \textbf{верхняя \word} the least upper bound;
}

% ----------------------------------------------------------------------------------
\dictchar{Д}
\dictentry{давать} {yield, а не give (\eg Equation (5) yields ...);}
\dictentry{делитель} {divisor;\\
  \textbf{наибольший общий \word} greatest common divisor;
}
\dictentry{диаметр} {diameter;\\
  \textbf{внешний \word} outside (а не outer) diameter, (o.d.);\\
  \textbf{внутренний \word} inside (а не inner) diameter, (i.d);
}
\dictentry{дифференцируемый} {differentiable;\\
  \textbf{дважды непрерывно \word} twice continuously differentiable;
  \textbf{\dot непрерывно \word\ $n$ раз} $n$ times differentiable;
  \textbf{\dot \word\ по $x$} $x$-differentiable (\cor differentiable
with respect to $x$);
  \textbf{\dot \word\ no Гато} G\^{a}teaux differentiable;
  \textbf{\dot слабо \word} weakly differentiable;
}
\dictentry{для этого} {to this end (\cor to do this);}
\dictentry{для краткости} {for brevity;}
\dictentry{для определенности} {for definiteness;}
\dictentry{для простоты} {for simplicity;}
\dictentry{доказывать} {prove (proved, proved (а не proven));\\
  \textbf{\dot по индукции} prove by induction;
  \textbf{\dot \word\ от противного} prove by contradiction;
}
\dictentry{допустимый} {admissible;\\
  \textbf{\word\ по Вольтерра} Volterra admissible;
}
\dictentry{друг друга} {one another, а не each other;}
\dictentry{достаточно (нар.)} {1) sufficiently (\eg For sufficiently small values of the parameter
  we have \ldots), 2) enough (\eg let $\varepsilon_\mu > 0$ be small enough to ensure that
  $\Theta(\varepsilon) \in (0, T]$. The value of this function is small enough.), 3. rather, fairly,
  quite;\\
  \textbf{\dot \word\ показать, что \ldots} it is sufficient to show that \ldots (\cor it suffices
  to show that \ldots);
}
\dictentry{достаточность} {sufficiency;}
\dictentry{достигать} {attain, reach, amount to;\\
  \textbf{\word\ максимума} attain a maximum;
}
\dictentry{дуга} {arc;\\
  \textbf{круговая \word} circular arc;
}

% ----------------------------------------------------------------------------------
\dictchar{Е}
\dictentry{если \ldots, тогда \ldots, в противном случае \ldots} {if \ldots then \ldots, otherwise \ldots;}
\dictentry{если и только если} {if and only if;}
\dictentry{если не оговорено противное} {if nothing is said to the contrary;}
\dictentry{единичный} {unit (\textit{see} матрица, оператор);}

% ----------------------------------------------------------------------------------
\dictchar{З}
\dictentry{завершать доказательство} {complete the proof;}
\dictentry{задача} {problem;\\
  \textbf{\dot ставить \word у} pose the problem;
  \textbf{\dot \word\ разрешима} the problem is solvable;
  \textbf{\dot разрешимость \word и} solvability of the problem;
  \textbf{\dot \word\ имеет единственное решение} the problem is uniquely solvable (\textit{or} the
  problem has a unique solution);
  \textbf{\dot постановка \word и} statement of the problem (\textit{see} корректная, некорректная);
  \textbf{\dot решать \word у} solve the problem;
  \textbf{\dot сводить \word у} reduce the problem to;
  \textbf{\dot \word\ сводится к} the problem is reducible to (\textit{or} the problem is reduced
  to);\\
  \textbf{корректно-поставленная} well-posed problem;\\
  \textbf{\word\ на собственные значения} eigenvalue problem;\\
  \textbf{краевая \word} boundary-value problem;\\
  \textbf{\word\ Коши} initial-value problem (Amer.), Cauchy problem (Europ.);\\
  \textbf{некорректно-поставленная \word} ill-posed problem;\\
  \textbf{обратная \word} inverse problem;\\
  \textbf{смешанная \word} initial boundary-value problem;\\
  \textbf{смешанная краевая \word} mixed boundary-value problem;\\
  \textbf{спектральная \word} eigenvalue problem;\\
  \textbf{\word\ Стефана} Stefan problem;\\
  \textbf{\word\ Штурма"--~Лиувилля} Sturm"--~Liouviile problem;
}
\dictentry{за исключением} {except, except for (\eg a vector with all elements zero except the
  $j$-th, which is unity);
}
\dictentry{замена} {change, replacement;\\
  \textbf{\dot \word\ переменных} variable change;
}
\dictentry{заметаемый} {spanned (\eg a hyperplane spanned by a maximum-dimension edge);}
\dictentry{замкнутость} {closedness;}
\dictentry{замкнутый} {closed;}
\dictentry{замыкание} {closure;}
\dictentry{значить} {imply, а не mean (\eg Equation (1) implies \ldots);}

% ----------------------------------------------------------------------------------
\dictchar{И}
\dictentry{из} {from;\\
  \textbf{\dot Из (1) видно, что \ldots} It is seen from (1) that \ldots, а не From (1) it is seen
  that \dots;
}
\dictentry{избыточно} {superfluous (\eg This equation is superfluous in system (1.3).);}
\dictentry{измерима по Лебегу} {Lebesgue measurable;}
\dictentry{инвариант} {invariant;\\
  \textbf{\word\ Римана} Riemann invariant;
}
\dictentry{инвариантный по отношению к оператору сдвига} {invariant under the translation operator;}
\dictentry{индекс} {index (\textit{pl.} indices, а не indexes);\\
  \textbf{верхний \word} superscript;\\
  \textbf{\word\ Ляпунова} Lyapunov index;\\
  \textbf{нижний \word} subscript;\\
  \textbf{центральный \word} central index;
}
\dictentry{интеграл} {integral;\\
  \textbf{\word\ Бохнера} Bochner integral;\\
  \textbf{\word\ Лебега} Lebesgue integral;\\
  \textbf{несобственный \word} improper integral;\\
  \textbf{повторный \word} repeated integral;\\
  \textbf{\word\ по контуру} contour integral;\\
  \textbf{\word\ столкновений} collision integral;
}
\dictentry{интегрирование} {integration;\\
  \textbf{\dot промежуток \word я} integration range;
  \textbf{\dot \word\ по частям} integration by parts;
  \textbf{почленное \word} term-by-term integration;
}
\dictentry{интегрируемый (который можно проинтегрировать)} {integrable;}
\dictentry{интегродифференцирование} {integrodifferentiation;}
\dictentry{интервал} {interval;\\
  \textbf{полуоткрытый \word} half-open interval;
}
\dictentry{искомый} {sought;}
\dictentry{исчерпывать} {exhaust (\eg The numbers $\nu_1,\ldots,\nu_k$ exhaust the set of
  eigenvalues of matrix $A$.);
}

% ----------------------------------------------------------------------------------
\dictchar{К}
\dictentry{касательность} {tangency;}
\dictentry{класс} {class;\\
  \textbf{\word\ эквивалентности} equivalence class;
}
\dictentry{коммутативность} {comnuitativity, а не commutivity;}
\dictentry{коммутатор} {commutator;}
\dictentry{коммутировать} {commute, а не commutate;}
\dictentry{конечномерный} {finite-dimensional;}
\dictentry{координата} {coordinate, а не co-ordinate;\\
  \textbf{\dot начало \word} the origin;\\
  \textbf{Декартовы \word ы} Cartesian coordinates;\\
  \textbf{криволинейные \word ы} curvilinear coordinates;\\
  \textbf{полярные \word ы} polar coordinates;\\
  \textbf{сферические \word ы} spherical coordinates;\\
  \textbf{цилиндрические \word ы} cylindrical coordinates;\\
  \textbf{эллипсоидальные \word ы} ellipsoidal coordinates;
}
\dictentry{корень} {root;\\
  \textbf{отличные (несовпадающие) \word и} distinct roots;\\
  \textbf{двукратный \word} double root;\\
  \textbf{простой \word} simple root;\\
  \textbf{трехкратный \word} triple root;
}
\dictentry{корректная постановка задачи} {well-posedness of the problem;}
\dictentry{кратное} {multiple;\\
  \textbf{\dot \word\ $2\pi$} a multiple of $2\pi$;
}
\dictentry{кратность} {multiplicity;\\
  \textbf{\dot \word\ корня} root multiplicity;\\
  \textbf{\word\ собственного значения} eigenvalue multiplicity;
}
\dictentry{критерий} {criterion (\textit{pl.} criteria);\\
  \textbf{\dot \word\ существования} criterion for (а не of) the existence (\textit{or} criterion
  for smth. to exist);
  \textbf{\word\ компактности оператора Римана-Лиувилля} a criterion for the Riemann-Liouville
  operator to be compact;\\
  \textbf{\word\ Арцелы} Arzel\`a criterion;\\
  \textbf{\word\ Дини} Dini criterion;\\
  \textbf{\word\ Рица} Riesz criterion;
}
\dictentry{кусочно-гладкая (-непрерывная, -постоянная \etc)} {piecewise-smooth (-continuous,
  -constant, etc.);
}

% ----------------------------------------------------------------------------------
\dictchar{Л}
\dictentry{легко} {easily;\\
  \textbf{\dot~\word\ видеть, что \ldots} it is easily seen that, а не can be easily seen that \ldots;
  \textbf{\dot~\word\ проверить} it is easy to verify (\eg This statement is easy to verify.);
  \textbf{\dot~\word~решаемая методом} easily solved by the method;
}
\dictentry{лемма} {lemma;\\
  \textbf{\word\ Гронуолла"--~Беллмана} Gronwall"--~Bellman lemma;
}

% ----------------------------------------------------------------------------------
\dictchar{М}
\dictentry{матрица} {matrix (\textit{pl.} matrices, а не matrixes);\\
  \textbf{\dot \word, обратная к $A$} an inverse of $A$;
  \textbf{\dot транспонированная \word} a transpose (не надо добавлять matrix);
  \textbf{\dot сопряженная \word} a conjugate;
  \textbf{\dot комплексно сопряженная \word} a complex conjugate;
  \textbf{\dot обращать \word у} invert a matrix;
  \textbf{\dot \word\ $n \times n$} an $n \times n$ matrix;\\

  \textbf{блочная \word} block matrix;\\
  \textbf{верхняя треугольная \word} upper triangular matrix;\\
  \textbf{единичная \word} identity (\textit{or} unit) matrix;\\
  \textbf{\word\ жесткости} stiffness matrix;\\
  \textbf{кососимметричная \word} skew-symmetric matrix;\\
  \textbf{плохо обусловленная \word} ill-conditioned matrix;\\
  \textbf{нижняя треугольная \word} lower-triangular matrix;\\
  \textbf{разреженная \word} sparse matrix, inflated matrix;\\
  \textbf{самосопряженная \word} self-adjoint matrix;\\
  \textbf{треугольная \word} triangular matrix;\\
  \textbf{хорошо обусловленная \word} well-conditioned matrix;
}
\dictentry{метод} {method, procedure;\\
  \textbf{итерационный \word\ Мозера} Moser iteration method;\\
  \textbf{\word\ последовательных приближений} method of successive approximations;\\
  \textbf{\word\ прямых} straight-line method;\\
  \textbf{\word\ разделения переменных} variable-separation method;
}
\dictentry{метрика} {metric;\\
  \textbf{\dot в \word е \ldots} in the metric of \ldots;\\
  \textbf{Хаусдорфова \word} Hausdorff metric;
}
\dictentry{многообразие} {manifold;\\
  \textbf{приводимое алгебраическое \word} reducible algebraic manifold;
}
\dictentry{множество} {set;\\
  \textbf{Борелево \word} Borel set;\\
  \textbf{отталкивающее \word} repelling set;\\
  \textbf{притягивающее \word} attracting set;
}
\dictentry{монотонно} {monotonically;\\
  \textbf{\dot \word\ возрастающий} monotonically increasing;
}
\dictentry{монотонность} {monotonicity;}
\dictentry{монотонный} {monotonic;}

\newpage
% ----------------------------------------------------------------------------------
\dictchar{Н}
\dictentry{на всем пространстве $H$} {on the whole of the space $H$;}
\dictentry{накладывать требования на \ldots} {impose requirements on \ldots;}
\dictentry{напоминать} {remind а не recall;}
\dictentry{не более} {not more than, at most (\eg the equation has at most three distinct roots);}
\dictentry{не менее} {not fewer than (with countable nouns), at least;}
\dictentry{невырожденный (оператор, матрица)} {nonsingular;}
\dictentry{не говоря о} {not to mention;}
\dictentry{недифференцируема} {nondifferentiable;}
\dictentry{недостаток места} {shortage of space (\eg shortage of space prevents us from reproducing
  this proof);
}
\dictentry{некорректная постановка задачи} {ill-posedness of the problem;}
\dictentry{не обращаясь к} {without referring to;}
\dictentry{необходимость} {necessity;}
\dictentry{не обязательно совпадают} {are not necessarily identical (\textit{or} not necessarily
  coincide);
}
\dictentry{неограниченность} {unboundedness;}
\dictentry{неограниченный} {unbounded;}
\dictentry{непрерывность} {continuity;\\
  \textbf{\dot \word\ в большом} continuity in the large;
  \textbf{\dot \word\ в малом} continuity in the small;
}
\dictentry{непрерывный} {continuous;\\
  \textbf{\dot \word\ слева (справа)} continuous to the left (right) (\textit{or} left (right)
  continuous);
  \textbf{\dot \word\ по Липшицу} Lipschits continuous;
}
\dictentry{неприводимый} {irreducible;}
\dictentry{непустой} {nonempty;}
\dictentry{неравенство} {inequality;\\
  \textbf{\word\ Гардинга} Garding inequality;\\
  \textbf{двустороннее \word} double-sided inequality;\\
  \textbf{\word\ Коши"--~Буняковского} Schwarz inequality;\\
  \textbf{\word\ Минковского} Minkovski inequality;\\
  \textbf{одностороннее \word} one-sided inequality;\\
  \textbf{строгое \word} strict inequality;\\
  \textbf{\word\ Юнга} Young inequality;
}
\dictentry{несколько слабее (сильнее \etc)} {somewhat weaker (stronger and so on);}
\dictentry{не является гладкой функцией от $t$} {fails to be smooth function of $t$;}
\dictentry{ноль} {zero, null;\\
  \textbf{\dot нули функции} zeros (а не zeroes) of the function;
}
\dictentry{нормаль} {normal;\\
  \textbf{внешняя \word} exterior normal;\\
  \textbf{внутренняя \word} interior normal;\\
  \textbf{единичная \word} unit normal;
}
\dictentry{нормировать} {normalize;\\
  \textbf{\dot \word\ на единицу} normalize to the unit;
}
\dictentry{носитель} {support;\\
  \textbf{\dot \word\ функции} support of the function;
}

% ----------------------------------------------------------------------------------
\dictchar{О}
\dictentry{обозначать} {denote;}
\dictentry{обозначение} {notation, а не notations;}
\dictentry{оболочка} {span (\eg the real linear span of the functions);}
\dictentry{образованы \ldots} { are formed of (а не from) \ldots;}
\dictentry{обратимость (матрицы)} {invertibility;}
\dictentry{обратимый} {invertible;}
\dictentry{обратный} {contrary (\textit{see} если не оговорено противное, задача, утверждение);\\
  \textbf{\dot если предположить \word ое} if we assume the contrary;
}
\dictentry{обращаться (становиться равным)} {turn to, become;\\
  \textbf{\dot \word\ в ноль} vanish, а не turn to zero;
}
\dictentry{общность} {generality;\\
  \textbf{\dot без потери \word и} with no loss of generality;
}
\dictentry{ограниченность} {boundedness;}
\dictentry{ограниченный} {bounded;\\
  \textbf{\dot \word\ по} bounded with respect to;
  \textbf{\dot \word\ сверху} bounded above;
  \textbf{\dot \word\ снизу} bounded below;
  \textbf{\dot не\word} unbounded;
}
\dictentry{однозначный} {single valued;}
\dictentry{однозначно определять} {to determine uniquely;}
\dictentry{однолистный} {one-sheeted;}
\dictentry{однородно по $t$} {uniformly with respect to $t$ (\textit{or} uniformly in $t$);}
\dictentry{ожидание} {expectation;\\
  \textbf{математическое \word} mathematical expectation (\textit{or} просто expectation);
}
\dictentry{означать} {imply, а не mean (\eg Lemma 1 implies that \ldots);}
\dictentry{окрестность} {neighborhood, а не vicinity;\\
  \textbf{\dot в окрестности начала координат} in the neighborhood of the origin;
}
\dictentry{оператор} {operator;\\
  \textbf{\word\ определенный на пространстве $L_{p_1}$ и действующий в $L_2$} operator defined on
  the space $L_{p_1}$ and acting on $L_2$;\\
  \textbf{\word\ Лапласа"--~Белтрами} Laplace"--~Beltrami equation;\\
  \textbf{\word\ Пуанкаре} Poincar\'e operator;\\
  \textbf{\word\ Римана"--~Лиувилля} Riemann"--~Liouville operator;\\
  \textbf{\word\ свертки} convolution operator;
}
\dictentry{опускать} {omit (\eg The arguments $x$ or $t$ of functions from $X^0_1$ will be sometimes
  omitted for brevity.), drop;
}
\dictentry{ослаблять} {weaken;}
\dictentry{оставаться в силе} {remain in force;}
\dictentry{остается неизменным} {remains unchanged, а не does not change;}
\dictentry{отдельно} {separately;\\
  \textbf{\dot рассматривать \word} consider separately (\eg We consider each of these cases
  separately.);
}
\dictentry{отличный (несовпадающий)} {distinct, а не different;\\
  \textbf{\dot всюду \word\ от нуля} everywhere nonzero;
}
\dictentry{отображать $B$ в $B_0$} {map $B$ into $B_0$;}
\dictentry{отображение} {mapping;\\
  \textbf{\word\ в себя} mapping into itself;\\
  \textbf{однозначное \word} one-to-one mapping;\\
  \textbf{сжимающее \word} contracting mapping;
}
\dictentry{отражение} {reflection;\\
  \textbf{\dot зеркальное \word\ относительно мнимой оси} mirror reflection in the imaginary axis;
}

% ----------------------------------------------------------------------------------
\dictchar{П}
\dictentry{переменная} {variable;\\
  \textbf{\dot замена \word ых} variable change;
}
\dictentry{перестановка} {permutation;}
\dictentry{переходить} {change [over] to, switch [over] to, transform to;\\
  \textbf{\dot \word\ в новую систему координат} transform to a new coordinate system;
}
\dictentry{периодичный с периодом $2\pi$ по аргументу $x$} {$2\pi$-periodic in the argument $x$;}
\dictentry{поверхность} {surface;\\
  \textbf{\word\ нулевого уровня} null-level surface;
}
\dictentry{подчеркивать} {stress, emphasize;}
\dictentry{позволять} {allow, enable, permit, make it possible;\\
  allow (permit, enable) smb. to do smth.;\\
  allow smth. to be done;\\
  Definition 1 makes it possible to compare the times when the level $R$ is reached.;
}
\dictentry{по крайней мере} {at least;\\
  \textbf{\word\ так же быстро как} at least as fast as;
}
\dictentry{покрытие} {covering;}
\dictentry{полагать $S=0.5$} {put $S=0.5$;}
\dictentry{положительно определенный} {positive definite, а не positively definite;}
\dictentry{полуограниченный} {semibounded;}
\dictentry{полупрямая} {half-line;}
\dictentry{полуцелый} {half-integer;}
\dictentry{понимать} {understand, mean;\\
  \textbf{\dot под решением уравнения (1) мы \word ем непрерывную функцию $f(x)$ такую-что \ldots} a
  solution of Eq. (1) is understood to be a continuous function $f(x)$ such that \ldots;
}
\dictentry{по определению} {by definition;}
\dictentry{посредством} {by virtue of;}
\dictentry{поступая как в \ldots} {proceeding as in \ldots;}
\dictentry{попарно различные} {pairwise-distinct;}
\dictentry{порядок} {order;\\
  \textbf{\dot величина \word а $\varepsilon$} a quantity of order $\varepsilon$;
}
\dictentry{последовательность} {sequence;\\
  \textbf{фундаментальная \word} fundamental sequence;
}
\dictentry{почленное} {term-by-term;}
\dictentry{почти всюду (непрерывен, компактен \etc)} {almost everywhere (continuous, compact, etc.);}
\dictentry{почти непрерывный (компактный \etc)} {almost continuous (compact etc.);}
\dictentry{поэтому} {therefore, а не thus;}
\dictentry{превосходить (быть больше)} {exceed;}
\dictentry{предел} {limit;\\
  \textbf{\dot \word\ слева (справа)} limit from the left (right);
}
\dictentry{предшествовать} {precede, to be first (\eg summation is first over the terms with plus
  sign and then over \ldots);
}
\dictentry{принимать} {take (\eg take the value of 0 and 1);\\
  \textbf{\dot \word\ значения $0, 1, 2, 3$} assume the values of $0, 1, 2, 3$;
  \textbf{\dot \word\ значения из $G$} run over $G$;
}
\dictentry{принцип} {principle;\\
  \textbf{\word\ максимума Понтрягина} Pontryagin maximum principle;\\
  \textbf{сильный \word\ максимума} strong maximum principle;
}
\dictentry{приравнивать} {equate;\\
  \textbf{\dot \word\ коэффициенты при \ldots} equate the coefficients at \ldots;
}
\dictentry{приходить к} {arrive at, а не to;}
\dictentry{проверка результатов (факта)} {verification of the results;}
\dictentry{продолжать} {continue, extend;\\
  \textbf{\dot \word\ $g_1, \ldots, g_k$ значением ноль на $(a, b)$} continue $g_1, \ldots, g_k$ by
  the value of zero to $(a, b)$;
}
\dictentry{продолжение} {continuation, extension;\\
  \textbf{\word\ Фридрихса} Friedrichs extension;
}
\dictentry{производная} {derivative;\\
  \textbf{\dot \word\ по $x$} derivative with respect to $x$;
  \textbf{\dot \word\ по времени} time derivative;\\
  \textbf{обыкновенная \word} ordinary derivative;\\
  \textbf{старшая \word} leading derivative;\\
  \textbf{\word\ Фреше} Frechet derivative;\\
  \textbf{частная \word} partial derivative;
}
\dictentry{промежуток} {range;\\
  \textbf{\word\ интегрирования} integration range:
}
\dictentry{пространство} {space;\\
  \textbf{\dot \word\ над числовым полем} space over the number field;\\
  \textbf{гильбертово \word} Hilbert space;\\
  \textbf{Евклидово \word} Euclidean space;\\
  \textbf{нуль-\word} null-space;\\
  \textbf{предгильбертово \word} pre-Hilbert space;
}
\dictentry{прямые вычисления} {direct calculations;}

% ----------------------------------------------------------------------------------
\dictchar{Р}
\dictentry{размерность} {dimension, а не dimensionality;}
\dictentry{разрешимость} {solvability;}
\dictentry{разрыв} {discontinuity;\\
  \textbf{\word\ первого рода} first-kind discontinuity;
}
\dictentry{раскладывать} {expand;\\
  \textbf{\word\ по степеням малого параметра} expand in powers of small parameter;
}
\dictentry{распространять получившееся соотношение на все $\nu \in L_2$} {extend the resulting
  relation to all $\nu \in L_2$;
}
\dictentry{рассеяние на} {scattering at;}
\dictentry{рассуждая, как и в [2]} {reasoning as in [2];}
\dictentry{рассуждения} {reasoning;\\
  \textbf{\dot \word, аналогичные вышеприведенным} reasoning similar to the foregoing;
}
\dictentry{резюмировать} {summarize;}
\dictentry{рекуррентный} {recurrent;}
\dictentry{pекуррентно} {recurrently;\\
  \textbf{\dot определять \word} define recurrently;
}
\dictentry{решение} {solution;\\
  \textbf{\dot искать \word\ в виде} seek the solution in the form;\\
  \textbf{нетривиальное \word} nontrivial solution;\\
  \textbf{сеточное \word} mesh solution;\\
  \textbf{\word\ Флоке} Floquet solution;
}

% ----------------------------------------------------------------------------------
\dictchar{С}
\dictentry{свертка} {convolution;}
\dictentry{свойство} {property;\\
  \textbf{\word\ Гёльдера} H\"older property;
}
\dictentry{связность} {connectivity;}
\dictentry{сетка} {mesh;\\
  \textbf{пространственно-временная \word} space-time mesh;
}
\dictentry{символ} {symbol;\\
  \textbf{\word\ Кронекера} Kronecker delta;
}
\dictentry{симметричный} {symmetric;\\
  \textbf{\dot \word\ относительно точки} symmetric about a point;
}
\dictentry{сингулярно возмущенный} {singularly perturbed;}
\dictentry{система} {system;\\
  \textbf{\word\ дифференциальных уравнений} differential equation system;\\
  \textbf{лабораторная \word\ координат} laboratory coordinate system;\\
  \textbf{полная \word} complete system;\\
  \textbf{\word\ Пфаффа} Pfaffian system;
}
\dictentry{сколь угодно малый} {arbitrarily small;}
\dictentry{слабо непрерывен (компактен \etc)} {weakly continuous (compact, etc.);}
\dictentry{следовательно} {hence, consequently;}
\dictentry{следствие (\textit{напр.} теоремы)} {corollary;}
\dictentry{следуя работе [1]} {following [1];}
\dictentry{случай, когда \ldots} {the case in which (а не when) \ldots;}
\dictentry{смысл} {sense, meaning;\\
  \textbf{\dot в некотором \word е} in a sense;
}
\dictentry{снабжать} {equip (\eg This equation is equipped with conventional boundary conditions.),
  furnish, endow (\eg We assume that $C^*(\Delta)$ is endowed with the weak norm.);
}
\dictentry{собственный вектор} {eigenvector;}
\dictentry{собственное пространство} {eigenspace;}
\dictentry{собственное число} {eigenvalue;\\
  \textbf{кратное \word} multiple eigenvalue;\\
  \textbf{простое \word} simple eigenvalue;
}
\dictentry{совместно с} {in conjunction with (\eg Theorems 1--3, in conjunction with (3.10) yield
  sufficient information for the investigation of \ldots);
}
\dictentry{совокупность} {totality, ensemble, aggregate, collection;\\
  \textbf{\dot \word\ аргументов} totality of arguments;
}
\dictentry{соответствующий} {corresponding, appropriate, suitable, proper;\\
  \textbf{\dot~\word им образом} properly;
  \textbf{\dot~собственный вектор, \word\ данному собственному значению} the eigenvector
  corresponding to this eigenvalue;
  \textbf{\dot~если сделать \word\ замену переменных, мы получим (1)} appropriate variable change
  yields (1);
}
\dictentry{сохранить члены уравнения} {retain the terms;}
\dictentry{с точностью до знака} {to within the sign;}
\dictentry{с точностью до константы} {to within an arbitrary constant;}
\dictentry{стремиться к \ldots при \ldots} {tend to \ldots for (with, when) \ldots;}
\dictentry{строго положительный} {strictly positive;}
\dictentry{сужение} {restriction;\\
 \textbf{\dot \word я функций из $\chi$ на $\mathbf{R}^n \times [s, T]$} restrictions of functions
 from $\chi$ to $\mathbf{R}^n \times [s, T]$;
}
\dictentry{сумма} {sum;\\
  \textbf{\dot \word\ по} the sum over;\\
  \textbf{прямая \word\ пространств} direct sum of spaces;
}
\dictentry{суммируемый (который можно просуммировать)} {summable;}
\dictentry{с учетом \ldots} {with \ldots  taken into account (\eg the set of eigenvalues with
  multiplicity taken into account);
}
\dictentry{счисление} {calculus (\eg operational calculus);}

% ----------------------------------------------------------------------------------
\dictchar{Т}
\dictentry{так (вводн. слово)} {for example, а не so;}
\dictentry{так же \ldots как и} {as \ldots as;}
\dictentry{таким образом} {thus, а не so;}
\dictentry{теорема} {theorem;\\
  \textbf{\dot \word\ о разрешимости системы (1)} theorem on solvability of system (1);\\
  \textbf{\word\ Асколи"--~Арцела} Ascoli"--~Arzel\`a theorem;\\
  \textbf{\word\ Банаха"--~Штейнхауса} Banach"--~Steinhaus theorem;\\
  \textbf{\word\ Биркхофа} Birkhoff theorem;\\
  \textbf{\word\ единственности} uniqueness theorem;\\
  \textbf{калибровочная \word} calibration theorem;\\
  \textbf{\word\ Леви"--~Деспланка} Levy"--~Desplank theorem;\\
  \textbf{\word\ Парсеваля} Parseval theorem;\\
  \textbf{\word\ Планчереля} Plancherel theorem;\\
  \textbf{\word\ сравнения} comparison theorem;\\
  \textbf{\word\ существования и единственности} existence and uniqueness theorem;\\
  \textbf{\word\ существования} existence theorem;\\
  \textbf{\word\ Фубини} Fubini's theorem;\\
  \textbf{\word\ Шоде о неподвижной точке} Schauder fixed-point theorem;
}
\dictentry{точка} {point;\\
  \textbf{несобственная \word} ideal point;\\
  \textbf{неподвижная \word\ (уравнения \etc)} fixed point;\\
  \textbf{предельная \word} limit (а не limiting) point;
}
\dictentry{точность} {accuracy, precision;\\
  \textbf{\dot с \word ю до} up to;
  \textbf{\dot с \word ю до знака} to within the sign;
}
\dictentry{точный} {exact;}
\dictentry{то, что они малы, показывается на примере уравнения (1)} {that they are small is shown by
  the example of Eq. (1).;
}
\dictentry{тройка} {triple;}

% ----------------------------------------------------------------------------------
\dictchar{У}
\dictentry{удовлетворять} {satisfy, meet;\\
  \textbf{\dot \word\ условиям} satisfy the conditions;
  \textbf{\dot \word\ требованиям} meet the requirements;
}
\dictentry{умножать на} {multiply by;\\
  \textbf{\dot \word\ слева} premultiply (\textit{or} left-multiply);
  \textbf{\dot \word\ справа} postmultiply (\textit{or} right-multiply);
}
\dictentry{уравнение} {equation;\\
  \textbf{\word\ Айри} Airy equation;\\
  \textbf{\word\ Брио"--~Буке} Briot"--~Bouquet equation;\\
  \textbf{\word\ в частных производных} partial differential equation;\\
  \textbf{\word\ Гельмгольца} Helmholtz equation;\\
  \textbf{Диофантово \word} Diophantine equation;\\
  \textbf{дифференциальные \word я Фукса} Fuchsian differential equations;\\
  \textbf{интегральное \word} integral equation;\\
  \textbf{\word\ Ито} Ito equation;\\
  \textbf{\word\ Кортевега-де Фриза} Korteweg-de Vries equation;\\
  \textbf{\word я Навье"--~Стокса} Navier"--~Stokes equations;\\
  \textbf{обыкновенное дифференциальное \word} ordinary differential equation;\\
  \textbf{\word\ Ора"--~Зоммерфельда} Orr"--~Sommerfeld equation;\\
  \textbf{\word\ Риккати} Riccati equation;\\
  \textbf{характеристическое \word} characteristic equation;
}
\dictentry{усиливать утверждение} {strengthen the assertion;}
\dictentry{условие} {condition, statement;\\
  \textbf{вышеупомянутое \word} foregoing condition;\\
  \textbf{достаточное \word} sufficient condition;\\
  \textbf{\word\ задачи} statement of the problem;\\
  \textbf{\word я Каратеодори} Carath\'eodory conditions;\\
  \textbf{краевое \word} boundary condition;\\
  \textbf{\word\ Липшица} Lipschitz condition;\\
  \textbf{начальное \word} initial condition;\\
  \textbf{необходимое \word} necessary condition;\\
  \textbf{необходимое и достаточное \word} necessary and sufficient condition;\\
  \textbf{однородное краевое \word} homogeneous boundary condition;\\
  \textbf{\word\ Рута"--~Гурвица} Routh"--~Hurwitz condition;\\
  \textbf{сильное \word} strong condition;\\
  \textbf{слабое \word} weak condition;\\
  \textbf{\word\ Хартмана} Hartman condition;
}
\dictentry{утверждать} {assert (\eg Lemma asserts that \ldots);}
\dictentry{утверждение} {assertion, proposition, statement;\\
  \textbf{обратное \word} converse assertion;
}

% ----------------------------------------------------------------------------------
\dictchar{Ф}
\dictentry{функция} {function;\\
  \textbf{\word\ Бесселя} Bessel function;\\
  \textbf{векторнозначная \word} vector-valued function;\\
  \textbf{гипергеометрическая \word} hypergeometric function;\\
  \textbf{$\delta$-\word\ Дирака} Dirac delta-function;\\
  \textbf{\word\ Лежандра} Legendre function;\\
  \textbf{матричнозначная \word} matrix-valued function;\\
  \textbf{однозначная \word} single-valued (\cor unambiguous) function;\\
  \textbf{сеточная \word} mesh function;\\
  \textbf{\word\ с ограниченной вариацией} a function of bounded variation;\\
  \textbf{\word\ Тичмарша} Titchmarsh function;\\
  \textbf{\word\ Тичмарша"--~Вейла} Titchmarsh-Weyl function;\\
  \textbf{\word\ Ханкеля} Hankel function;\\
  \textbf{\word\ Хэвисайда} Heaviside unit function;
}

% ----------------------------------------------------------------------------------
\dictchar{Ц}
\dictentry{цепь} {chain;\\
  \textbf{вложенная \word\ Маркова} imbedded Markov chain;
}

% ----------------------------------------------------------------------------------
\dictchar{Ч}
\dictentry{частный случай} {special case, а не particular case;}
\dictentry{часть} {side;\\
  \textbf{\dot в левой \word и} in the left side of (\cor on the left of);
  \textbf{\dot главная \word\ интеграла} principal part of the integral;\\
  \textbf{левая \word} left (\cor left-hand) side;\\
  \textbf{правая \word} right (\cor right-hand) side;\\
  \textbf{целая \word} integral part (\cor the greatest integer in);
}
\dictentry{через} {in terms of, through, via;\\
  \textbf{\dot выражать через \ldots}  express in terms of \ldots;
}
\dictentry{черта} {bar;\\
  \textbf{\dot Черта над буквой обозначает комплексно"=сопряжённую величину.}
  The bar over a letter indicates a complex-conjugate.;
}
\dictentry{число} {number;\\
  \textbf{число обусловленности} conditioning number;
}

% ----------------------------------------------------------------------------------
\dictchar{Ш}
\dictentry{шаблон} {stencil;}
\dictentry{штрих} {prime;\\
  \textbf{\dot \word\ у переменной обозначает \ldots}\ a prime on a variable indicates \ldots;
  \textbf{\dot~величина со \word ом} primed quantity;
}
\dictentry{шар} {ball;\\
  \textbf{единичный \word} unit ball;
}

\end{document}

% vim: et:sw=2:textwidth=100:ft=tex:
